% General packages
\usepackage[utf8]{inputenc} % utf8 input good for non-ascii characters
\usepackage[T1]{fontenc} % Good for hyphenating non-ascii characters
\usepackage{lmodern}
\usepackage[british]{babel}
\usepackage[nottoc]{tocbibind} % Includes bibliography in table of contents, but not the table of contents itself
\usepackage[figuresright]{rotating} % Allows for sideways figures and tables
\usepackage[hidelinks,pdfusetitle]{hyperref} % hidelinks removes the boxes around links when viewed on screen, pdfusetitle option includes the thesis title and author in the PDF metadata

\usepackage[super,sort&compress,comma]{natbib}
\bibliographystyle{rsc}

% Glossaries / list of acronyms
%\usepackage[toc,automake,nopostdot,numberline,indexonlyfirst,acronym]{glossaries}
\usepackage[toc,acronym,nopostdot]{glossaries}
\makenoidxglossaries
\loadglsentries{frontmatter/acronyms}


% Small caps for chapter and section headings
\usepackage{titlesec}
\titleformat{\chapter}[display]{\sc\LARGE}{Chapter \thechapter}{1em}{}
\titleformat{\section}[block]{\sc\Large}{\thesection}{1em}{}
\titleformat{\subsection}[block]{\sc}{\thesubsection}{1em}{}
\titleformat{\subsubsection}[block]{\sc}{\thesubsubsection}{1em}{}

% Nicer appearance / typesetting
\usepackage[shrink=10,stretch=10]{microtype}
% We define a command to disable and enable protrusion, used for the tables of contents
\makeatletter
\@ifpackageloaded{microtype}{%
	\providecommand{\disableprotrusion}{\microtypesetup{protrusion=false}}
	\providecommand{\enableprotrusion}{\microtypesetup{protrusion=true}}
}{%
	\providecommand{\disableprotrusion}{}
	\providecommand{\enableprotrusion}{}
}
\makeatother

\usepackage{booktabs} % Better tables

% These make paragraphs not indented, and separate consecutive paragraphs
\setlength{\parindent}{0pt}
\setlength{\parskip}{0.5em plus 3pt minus 3pt}

% Format captions
\usepackage[margin=15mm,hang,font=sf,labelfont=bf]{caption}
\usepackage{subcaption}
\usepackage{perpage} % Allows to reset counters on each page
\MakePerPage{footnote} % Resets footnote counters on each page


% For convenience
% The folder in which images are stored for this project.
% If this is enabled, the folder doesn't need to be specified in each
% call to \includegraphics, i.e
%   \includegraphics{picturename}
% rather than
%   \includegraphics{img/picturename}
\graphicspath{{figs/}}

% In align*, use this to number a particular line
% Rather than using align, and \nonumber-ing every other line
\newcommand\numberthis{\addtocounter{equation}{1}\tag{\theequation}}

% Provide itemize and enumerate without extra spacing
\newenvironment{itemize*}%
{\begin{itemize}%
	\setlength{\itemsep}{0pt}%
	\setlength{\parskip}{0pt}}%
{\end{itemize}}
\newenvironment{enumerate*}%
{\begin{enumerate}%
	\setlength{\itemsep}{0pt}%
	\setlength{\parskip}{0pt}}%
{\end{enumerate}}


% Better maths support
\usepackage{amsmath,amssymb}
\numberwithin{equation}{section}
\allowdisplaybreaks % Allow display equations to break over different pages
\usepackage{amssymb}
\usepackage{bm} % More bold maths symbols
\usepackage{bbm} % More blackboard bold characters, via \mathbbm. Mostly for blackboard bold 1
\usepackage{xfrac} % Nice small fractions
\usepackage{array} % Allows us to define custom column types for tables
\newcolumntype{L}{>{$}l<{$}} % a left aligned maths column type

\usepackage{cleveref} % Add automatic reference type text via \cref command


% Some useful physics packages
%\usepackage[italic]{hepnames} % Add particle name macros (e.g. \PBs)
%\usepackage{braket} % Adds Dirac bra-ket notation
%\usepackage{slashed} % Adds Dirac slash notation
\usepackage{siunitx} % Add support for units
\sisetup{
	separate-uncertainty, % uncertainties with +- symbol, 
	range-phrase = --, % ranges with dash
	range-units = single % only write unit once
} 
\DeclareSIUnit\fb{\femto\barn}
